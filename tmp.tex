% !Mode:: "TeX:UTF-8"
%
\documentclass[fontset=fandol,toc=true,type=doctor,stage=opening,campus=harbin]{hithesisart}
% 此处选项中不要有空格
%%%%%%%%%%%%%%%%%%%%%%%%%%%%%%%%%%%%%%%%%%%%%%%%%%%%%%%%%%%%%%%%%%%%%%%%%%%%%%%%
% 必填选项
% type=doctor|master|bachelor
% stage=opening|midterm
%%%%%%%%%%%%%%%%%%%%%%%%%%%%%%%%%%%%%%%%%%%%%%%%%%%%%%%%%%%%%%%%%%%%%%%%%%%%%%%%
% 选填选项(选填选项的缺省值已经尽可能满足了大多数需求,除非明确知道自己有什么
% 需求)
% campus=shenzhen|weihai|harbin
%   含义:校区选项,默认harbin
% fontset=windows|mac|ubuntu|fandol
%   含义:前三个对应各自系统,fandol是开源字体。
%%%%%%%%%%%%%%%%%%%%%%%%%%%%%%%%%%%%%%%%%%%%%%%%%%%%%%%%%%%%%%%%%%%%%%%%%%%%%%%%
\renewcommand{\thefigure}{\arabic{section}-\arabic{figure}}%使图编号为 7-1 的格式 %\protect{~}
\renewcommand{\thetable}{\arabic{section}-\arabic{table}}%使表编号为 7-1 的格式
\renewcommand{\theequation}{\arabic{section}-\arabic{equation}}%使表编号为 7-1 的格式
% \numberwithin{equation}{section}%公式按章节编号
\newcommand{\subsubsubsection}[1]{\paragraph{#1}\mbox{}}
\setcounter{secnumdepth}{4} % how many sectioning levels to assign numbers to
% \setcounter{tocdepth}{4} % how many sectioning levels to show in ToC
\usepackage{tocloft}
\setlength{\cftbeforesecskip}{2pt}
\usepackage{float}
\usepackage{caption}
\usepackage{multirow}
\usepackage{rotating}

\pagestyle{mypagestyle}
\begin{document}
% \fancyhead[C]{哈尔滨工业大学博士学位论文开题报告}
% \fancyhead[C]{\songti\xiaowu 哈尔滨工业大学博士学位论文开题报告}
% \makeatletter %双线页眉
% \def\headrule{{\if@fancyplain\let\headrulewidth\plainheadrulewidth\fi%
% \hrule\@height 0.5pt \@width\headwidth\vskip1pt%上面线为1pt粗
% \hrule\@height 2.25pt\@width\headwidth  %下面0.5pt粗
% \vskip-2\headrulewidth\vskip-1pt}      %两条线的距离1pt
% \vspace{6mm}}     %双线与下面正文之间的垂直间距
% \makeatother

\captionsetup{font={small}}

% \newCJKfontfamily\sonti{FZCuSong-B09S}[BoldFont=FZCuSong-B09S]
\hitsetup{
  ctitlecover={基于方法研究},%放在封面中使用,自由断行
  caffil={电子与信息工程学院},
  csubject={信息与通信工程},
  cauthor={张三},
  cstudentid={123456},
  csupervisor={李四教授},
  % 日期自动使用当前时间,若需指定按如下方式修改:
  cdate={2022年9月},
}
\makecover
{\centering{\tableofcontents}}
\thispagestyle{empty}
\setcounter{page}{0}
\newpage
% 深圳博士开题报告 -----------------------------------------------

\section{课题来源及研究的目的和意义}
\subsection{课题来源}
本课题来源于国家自然科学基金项目。
\subsection{课题研究的目的和意义}
\section{国内外在该方向的研究现状及分析}
\subsection{某研究现状}
\subsection{国内外文献综述的简析}
\section{学位论文的主要研究内容、实施方案及其可行性论证}
\subsection{主要研究内容}
\subsection{实施方案}
\subsubsection{技术原理}
\subsection{可行性论证}
\subsubsection{实验数据保障}
\subsubsection{研究方案的合理性}
\section{前期的理论研究与试验论证工作的结果}
\subsection{基于算法结果}
\subsubsection{实验分析}
\section{论文进度安排,预期达到的目标}
\subsection{进度安排}
2020年9月-2021年8月,参加并完成博士研究生课程的学习,阅读相关的资料与文献,确定研究课题的基本方向。
\subsection{预期达到的目标}
\section{学位论文预期创新点}
本论文预期创新点如下:
\section{为完成课题已具备和所需的条件、外协计划及经费}
本论文的研究方向在实验室具有较充分的研究基础,
\section{预计研究过程中可能遇到的困难、问题,以及解决的途径}
\noindent
(1)存在的问题:
\section{主要参考文献}
\section{论文发表情况}
\renewcommand{\labelenumi}{[\arabic{enumi}]}
\begin{enumerate}
  \item 1
  \item 2
  \item 3
\end{enumerate}
\end{document}